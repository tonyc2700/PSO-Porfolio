\chapter{Evaluation and Conclusion}
This chapter summarises and discusses the tasks accomplished, providing a critical evaluation of the work that was done and suggesting future development.

  \section{Evaluation} % (fold)
  \label{sec:evaluation}
  Looking back at the project goals and requirements, it is possible to say that almost all of them were successfully achieved. 
  \begin{itemize}
    \item \textbf{Understanding the principles behind the Particles Swarm Optimisation algorithm and designing an appropriate expansion to solve the portfolio optimisation problem.} \\
    Blahh
    \item \textbf{Testing the resulting application on a dataset from a real-world domain} \\
    \item \textbf{Perform experiments on the dataset to assess the effectiveness of the modified algorithm while making necessary changes to the parameter settings for a better optimisation and performance} \\
    \item \textbf{Something else} \\
  \end{itemize}
  % section evaluation (end)

  \section{Problems encountered} % (fold)
  \label{sec:problems_encountered}
    \begin{itemize}
      \item Haskell being strong typed
      \item Results are too small (to good!!) for good presentation
      \item A lot of new financial information had to be learnt
      \item This is bigger and more important than anticipated, 
    \end{itemize}
  % section problems_encountered (end)

  \section{Discussion} % (fold)
  \label{sec:discussion}
  When I first discussed this project with my supervisor Dr. Wei Pang, I had no idea how vast the scope was, I had no idea how big this topic really is. I believe I could do a 4 PHD in this topic and I would not even begin to break the ice. I have had to learn huge amounts of new things. I have never studied any sort of finance or economics course. I have to say, watching many hours of online lectures from various universities helped an incredible amount in understanding complex concepts, as well as basic ones, in financial risk management.

  This project has given me a renewed motivation to continue studying as well as directions of what to study. I believe the two things I need to learn more about if I want to continue improving this system are, more knowledge on asset risk measure and real-time processing. 
  % section discussion (end)

  \section{Conclusion} % (fold)
  \label{sec:conclusion}
  In conclusion, this project could be rated as a success. The research was an enlightening introduction into the world of evolutionary algorithms and their applications, specially the principles of the Particle Swarm Optimisation algorithm and how it can be applied in the real-world domain.

  Most of the primary and secondary project goals were achieved. The modified PSO algorithm incorporates new principles for solving the portfolio selection problem and lots of designs have been stated on how to further improve this system. Dues to time and other commitment constraints, not all were implemented, but clear designs have been made and shown. The algorithm was tested on financial data and the results were satisfactory.

  Many challenges were faced during the design and implementation stages of the improved algorithm; some alternative paths considered, Regardless, this research project has given me a valuable insight into the world of evolutionary algorithms, their principles and applications. The results gained from the testing have given me a more insight knowledge on the algorithm, it has taught me new facts about PSo and confirmed some previous known facts too.

  There are plenty of opportunities for future development, I believe that this research project has opened a door for me to create something which I believe might change the way investment companies deal with portfolio selection.

  
  % section conclusion (end)

