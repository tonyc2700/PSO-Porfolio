\chapter{Introduction\label{chap:intro}}
This section provides an overview of the project and explains the basic principles of the initial approach and ideas for expansion. It contains a list of primary and secondary goals as well as motivation for carrying out this research.

  \section{Overview} % (fold)
  \label{sec:overview}
  Swarm Algorithms that belong to the Swarm Intelligence Systems were inspired by the behaviour of ant colonies, bird flocking, animal herding, bacterial growth, and fish schooling. As flocks of birds, the artificial swarm is able find an optimal location through communication with local agents as well as the environment. Swarm Algorithms have proven to be effective in providing ways to find global optima in potentially difficult search spaces. The techniques used in Swarm Intelligence Systems closely related to artificial intelligence and are often applied to simulations, robotics and optimisation problems. 

  The Particle Swarm Optimisation algorithm belongs to the field of Swarm Intelligence Systems and optimizes a problem by having a population of candidate solutions, called particles, moving these particles around in the search-space according to simple mathematical formulae over the particle's position and velocity, shown in Equation~(\ref{eq:pso}). Each particle's movement is influenced by its local best known position but, is also guided toward the best known positions in the search-space, which are updated as better positions are found by other particles. Originally, PSO was designed to simulate bird flocking behaviour \cite{pso}, it was later discovered that it would be used to find optimal positions, such as a swarm of birds finding food in a large city.
  % section overview (end)

  \section{Motivation} % (fold)
  \label{sec:motivation}
  % section motivation (end)

  \section{Primary Goals} % (fold)
  \label{sec:primary_goals}
  A list of primary goals which need to be completed in order to classify the research project as a success.
  \begin{itemize}
    \item Understand the principles behind Particle Swarm optimisation.
    \item Design an appropriate expansion to solve the portfolio selection problem.
    \item Understand the principles begin portfolio risk measure.
    \item Test the resulting application on a dataset from a real-world domain.
  \end{itemize}
  % section primary_goals (end)

  \section{Secondary Goals} % (fold)
  \label{sec:secondary_goals}
  Secondary goals could be added to the project depending on time-constraints and success of completing the primary goals.
  \begin{itemize}
    \item Compare my implementation to that of an investment manages. 
    \item Write a simple GUI for the application
  \end{itemize}
