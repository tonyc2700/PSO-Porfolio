\chapter*{Appendix B: Maintenance Manual}
The source code of the PSO and portfolio optimisation (together with an example asset file) could be found in the `source' folder in project submission.

\section*{Instructions on how to compile and run the algorithm:} 
  Makes sure that the source Haskell files (portOpt.hs, PSO.hs) are located in the same folder as the asset data files you wish to use for the portfolio.

  \begin{enumerate}
    \item Open a terminal and move to the folder containing the source files, ie the Haskell code.
    \item Type the following command into the terminal and press enter:
  \suspend{enumerate}
  \begin{center} \% ghc -O2 --make portOpt.hs -threaded -rtsopts \end{center}
  \resume{enumerate}
    \item You have now created an executable file, now type into the terminal the following and press enter:
  \suspend{enumerate}
  \begin{center} \% ./portOpt \end{center}
  \resume{enumerate}
    \item The program will start in the same terminal window.
  \end{enumerate}
  The User Manual provides a guide for using the program. \\

\section*{Organisation of System Files, including directory structures} 

\section*{Space and Memory Requirements}
  2 GB of RAM, 100 MB of space would be recommended.

\section*{List of source code files, with a summary of their role}
  \begin{tabular}{|l|l|}
    \hline
    File & Description \\
    \hline
    portOpt.hs & Portfolio optimisation implementation using PSO \\
    \hline
    PSO.hs & Main PSO optimisation module \\
    \hline
  \end{tabular}

\section*{Program Flow} 
  This section provides a general description of the program flow from a technical side. \\
  \textbf{Initialisation} \\

  \textbf{Optimisation} \\
  
  \textbf{Output} \\


\section*{Changing Parameters} 
