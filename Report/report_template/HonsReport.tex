%%
%%  HonsReport.tex - Latex Sampler...
%%
%%  Dave Ritchie, October 2002
%%
%%-----------------------------------------------------------------------
%%
%%  This example uses the "mscthesis" style, which assumes you have the
%%  local Latex environment defined. To setup Latex on raven, for example:
%%
%%     raven% setup tex
%%
%%  To typeset this script:
%%
%%     latex HonsReport.tex
%%
%%  To view the result:
%%
%%     xdvi HonsReport.dvi
%%
%%  To print:
%%
%%     dvi2ps -o HonsReport.ps HonsReport.dvi; lpr HonsReport.ps
%%
%%  etc...
%%
%%  The class and style files "mscthesis.cls" and "mscthesis.sty" are
%%  taken from the curent directory, or from 
%%
%%    /local/tex/teTeX/texmf.local/tex/latex/mscthesis.cls
%%
%%  Hence these could be copied and customised if necessary.
%%
%%  The local Latex Guide is at:
%%
%%    http://docs.csd.abdn.ac.uk/LaTeX/
%%
%%-----------------------------------------------------------------------
\documentclass[a4paper,12pt]{mscthesis}

\RequirePackage{float}
\RequirePackage{ifthen}
\RequirePackage{epsfig}
\RequirePackage{amsmath}
\RequirePackage{tabularx}
\RequirePackage{subfigure}
\RequirePackage{supertabular}


\begin{document}

\title{How I Did It!}

\author{H. Ackermann}

\abstract{This report describes a revolutionary breakthrough in
          software engineering. 
          An entirely new programming paradigm is presented
          in which the engineer writes a vague description of
          the requirements and the software system generates
          executable code using the {\em ``I thought that's what
          you meant principle.''} 
          Artificial Intelligence techniques are used to 
          sample fragments of existing code from commercial programs,
          and a low level parser is used to combine these fragments
          into an executable program. 
          In 90\% of cases, the given requirements are so vague that 
          a panel of expert agree that the generated code 
          could indeed be said to satisfy at least 70\% of the
          requirement specification. 
          Users of new the system found it to be almost indistinguishable
          from many existing desktop products.
          One expert remarked that ``the new system could save many people 
          thousands of dollars - they would never again need to buy
          expensive and often unnecessary software.''
}

%%-----------------------------------------------------------------------

%%
%%  Create the first page(s)
%% 

\maketitle

\acknowledgements{
I would like to thank my family, my colleagues, and everyone else
who knows me.}

%%-----------------------------------------------------------------------

%%
%% Do Honours Projects require a declaration ??
%% 
%% \declaration

%%-----------------------------------------------------------------------

%%
%%  Make the rest of the "front matter"
%%

\tableofcontents

%% \listoffigures

%% \listoftables

\clearpage

%%-----------------------------------------------------------------------

%%  we probably want wide line spacing for Reports

\widespacing


\chapter{Introduction (Why Its Important/Worthwhile)}

In this report I will treat my grave subject with deadly
seriousness. There may be doubters \cite{wilder:1985},
but this document will demonstrate the error of their ways,
and explain in full how I achieved my {\em remarkable results}.

The rest of this Report is divided into Chapters and SubSections. 
The Chapter Headings and some of the Subsection Headings might
actually make sense. The rest of this stuff is of course 
complete rubbish.

\section{Project Overview}

\section{Project Constraints}

\chapter{Background (Where I Started From)}

\section{Existing Code Generators}

\section{A New Paradigm for Code Generation}

\subsection{The Fractality Principle}

\subsection{Distinguishing Randomness and Complexity}

\subsection{Generating Random Code But Nice-Looking Code}

\chapter{Methods (How I Did It)}

\chapter{Results (How Good It Is)}

\chapter{Conclusions (What I Have Achieved)}

\subsection{Achievements}
\subsection{Future Work}
\subsection{Summary}
%%
%%  how to include another source file (discussion.tex)
%%
%% \include{discussion}

%%-----------------------------------------------------------------------

%% rename chapters as appendices

\appendix

\chapter{An Appendix}

%%-----------------------------------------------------------------------

%%
%%  Need an HonsReport.bib file to get a bibliography
%%

\bibliographystyle{unsrt}

\addcontentsline{toc}{chapter}{\numberline{}Bibliography}
\bibliography{HonsReport}

\end{document}
