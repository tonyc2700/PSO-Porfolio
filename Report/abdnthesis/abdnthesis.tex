\documentclass[bsc,doublespace]{abdnthesis}

%% For citations, I would recommend natbib for its
%% flexibility, particularly when named citation styles are used, but
%% it also has useful features for plain and those of that ilk.
%% The natbib package gives you the following definitons
%% that extend the simple \cite:
%   \citet{key} ==>>                Jones et al. (1990)
%   \citet*{key} ==>>               Jones, Baker, and Smith (1990)
%   \citep{key} ==>>                (Jones et al., 1990)
%   \citep*{key} ==>>               (Jones, Baker, and Smith, 1990)
%   \citep[chap. 2]{key} ==>>       (Jones et al., 1990, chap. 2)
%   \citep[e.g.][]{key} ==>>        (e.g. Jones et al., 1990)
%   \citep[e.g.][p. 32]{key} ==>>   (e.g. Jones et al., p. 32)
%   \citeauthor{key} ==>>           Jones et al.
%   \citeauthor*{key} ==>>          Jones, Baker, and Smith
%   \citeyear{key} ==>>             1990

% \usepackage[round,colon,authoryear]{natbib}
% \setlength{\bibsep}{0pt}
% \bibliographystyle{apalike}

\usepackage[T1]{fontenc}
\usepackage{hyperref}
\usepackage{amsmath,amsthm}
\usepackage{color} 
\usepackage{graphicx}
\usepackage{float}
\usepackage{algorithm2e}
\usepackage{tikz}
\usepackage{array}
\usepackage{setspace}
\usetikzlibrary{decorations.pathmorphing}
\usetikzlibrary{decorations.fractals}
\usepackage{mdwlist}
\usepackage{fancyvrb}
\DefineVerbatimEnvironment{code}{Verbatim}{fontsize=\small}
\DefineVerbatimEnvironment{example}{Verbatim}{fontsize=\small}
\newcommand{\ignore}[1]{}

\newtheorem*{defn}{Definition}
\newtheorem*{thm}{Theorem}
\newtheorem*{exa}{Example}
\newtheorem*{no}{Note}

\graphicspath{ {./Figures/} }

\title{Particle Swarm Optimisation \\ to solve the Portfolio Selection Problem \\ in a Function Based Environment \\ (30 Credit Course)}
\author{Anthony Sergio Chapman}
\Supervisor{Dr. Wei Pang}
% IMO this is a bit silly, but some like to include these. To remove,
% delete this declaration and remove the option from the
% \documentclass definition above.
%\qualifications{PhD, Computer Science, University College London, 1997\\%            
%BEng (Hons.) Electrical and Electronic Engineering, The University of Wales, Swansea, 1992}
\school{Department of Computing Science}

%%%% In the final submission of a thesis, this should only be the year
%%%% of submission.  However, it is useful to use \date{\today} for drafts so that
%%%% they don't get mixed up.
    
\date{2014}

%% It is useful to split the document up as chapters and include
%% them.  LaTeX will sort out all the numbering and cross-referencing
%% for you --- if you run it enough times!

%% If you want to include only a couple of chapters then use the
%% \includeonly{} command with a list of the file/chapter names that
%% you wish to include.  NB, this must be in the preamble.

% \includeonly{introduction,faq,chap1}

\def\sfthing#1#2{\def#1{\mbox{{\small\normalfont\sffamily #2}}}}

\sfthing{\PP}{P}
\sfthing{\FF}{F}

%% This will make sure that all cross-references are correct (including
%% references to those file not included) but will produce a dvi
%% file with only those files/chapters you specify included.

\begin{document}

%%%% Create the title page and standard declaration.

\maketitle
\makedeclaration

%%%% Then the abstract and acknowledgements

\begin{abstract}
  This research project aims to investigate how the Particle Swarm Optimisation (PSO) algorithm could be applied for multi-dimensional optimisation in a real-world domain, namely finding an optimal portfolio to invest in, given a set of assets and their expected returns. The Particle Swarm Optimisation algorithm, created by James Kennedy and Russell Eberhart, belongs to the field of Swarm Intelligence. It uses a collection of particles (a swarm) to explore a given search space for an optimal value. To be able to solve the portfolio optimisation problem, a way to measure and compare different portfolio had to be created. Using two-sided coherent risk measure, created by Zhiping Chen and Yi Wang, and concepts from mathematics, measure theory, the portfolio optimisation problem turned into a function which the PSO algorithm is able to optimise. 

  The main goal of this project is to create a reliable method for computing and comparing a portfolio's success or failure in order for the algorithm to be able to find an optimal one. Secondary goals such as experimenting and expanding the PSO algorithm were also achieved. Finally, the effectiveness of the resulting approach was assessed through experimentation and different parameter settings. The report shows the background research, design, implementation, testing and experimentation with the expanded Particle Swarm Optimisation algorithm. 
\end{abstract}

\begin{acknowledgements}
  Firstly, I would like to thank my supervisors, Dr. Wei Pang, for providing help and support throughout the whole project. His advice, enthusiasm and knowledge of the field have always pointed me in the right direction with the research. 

  Secondly, I would also like to thank Dr. Fernando Rubio, from U.C.M., for proving additional support for expanding the Haskell PSO implementation and finding time to answer my inquiries. 

  Lastly, thanks to all my 205 buddies for providing moral support and advise at all hours of the day.
\end{acknowledgements}

%%%% It should have a table of contents, but delete the other two as
%%%% necessary.

\tableofcontents
\listoftables
\listoffigures

\include{chap1-intro}
\include{chap2-background}
\include{chap3-related}
\include{chap4-problem}
\include{chap5-require}
\include{chap6-risk}
\include{chap7-methodology}
\include{chap8-design}
\include{chap9-test}
\include{chap10-future}
\include{chap11}

\bibliographystyle{plain}
\bibliography{myref}

\appendix
\include{chap12-user}
\chapter*{Appendix B: Maintenance Manual}
The source code of the PSO and portfolio optimisation (together with an example asset file) could be found in the `source' folder in project submission.

\section*{Instructions on how to compile and run the algorithm:} 
  Makes sure that the source Haskell files (portOpt.hs, PSO.hs) are located in the same folder as the asset data files you wish to use for the portfolio.

  \begin{enumerate}
    \item Open a terminal and move to the folder containing the source files, ie the Haskell code.
    \item Type the following command into the terminal and press enter:
  \suspend{enumerate}
  \begin{center} \% ghc -O2 --make portOpt.hs -threaded -rtsopts \end{center}
  \resume{enumerate}
    \item You have now created an executable file, now type into the terminal the following and press enter:
  \suspend{enumerate}
  \begin{center} \% ./portOpt \end{center}
  \resume{enumerate}
    \item The program will start in the same terminal window.
  \end{enumerate}
  The User Manual provides a guide for using the program. \\

\section*{Organisation of System Files, including directory structures} 

\section*{Space and Memory Requirements}
  2 GB of RAM, 100 MB of space would be recommended.

\section*{List of source code files, with a summary of their role}
  \begin{tabular}{|l|l|}
    \hline
    File & Description \\
    \hline
    portOpt.hs & Portfolio optimisation implementation using PSO \\
    \hline
    PSO.hs & Main PSO optimisation module \\
    \hline
  \end{tabular}

\section*{Program Flow} 
  This section provides a general description of the program flow from a technical side. \\
  \textbf{Initialisation} \\

  \textbf{Optimisation} \\
  
  \textbf{Output} \\


\section*{Changing Parameters} 


\end{document}
